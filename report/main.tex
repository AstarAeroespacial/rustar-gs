\documentclass[a4paper,11pt]{article}

% --- Language and font ---
\usepackage[spanish]{babel}
\usepackage[utf8]{inputenc}
\usepackage[T1]{fontenc}
\usepackage{mathptmx} % Times New Roman-like

% --- Layout and spacing ---
\usepackage{geometry}
\geometry{margin=2.5cm}
\usepackage{setspace}
\onehalfspacing % Line spacing 1.5

\usepackage{parskip}
\setlength{\parskip}{0pt} % No space after paragraph
\setlength{\parindent}{0pt} % Optional: no indent

% --- Graphics ---
\usepackage{graphicx}

% --- Title formatting ---
\usepackage{titlesec}
\titleformat{\section}
  {\normalfont\fontsize{16}{18}\bfseries}{\thesection}{1em}{}
\titleformat{\subsection}
  {\normalfont\fontsize{14}{16}\bfseries}{\thesubsection}{1em}{}
\renewcommand{\normalsize}{\fontsize{11}{13}\selectfont}

\titlespacing*{\section}{0pt}{0pt}{0pt}
\titlespacing*{\subsection}{0pt}{0pt}{0pt}

% --- Headers and footers ---
\usepackage{fancyhdr}
\pagestyle{fancy}
\fancyhf{}
\fancyhead[C]{\textit{RUSTAR - Software para estación terrena}}
\fancyfoot[L]{Davies, Lazcano, Medone Sabatini, Bohórquez}
\fancyfoot[R]{\thepage}

% --- Document begins ---
\begin{document}

% --- Cover Page ---
\begin{titlepage}
\begin{center}

\includegraphics[width=7cm]{logo} \\[2cm]

\textsc{\Large Trabajo Profesional} \\[0.5cm]
{\huge \bfseries RUSTAR \\[0.3cm]
Software para estación terrena} \\[1cm]

\rule{\linewidth}{0.4pt} \\[1cm]

{\large \today} \\[2cm]

\hspace*{-1.5cm}
\setstretch{1.5}
\begin{tabular}{ | l | l | }
  \hline
  \textbf{Integrantes} & \textbf{Padrón} \\ \hline
  Davies, Alen & 107084 \\ \hline
  Lazcano, Luca & 107044 \\ \hline
  Medone Sabatini, Juan Ignacio & 103878 \\ \hline
  Bohórquez, Rubén & 109442 \\ \hline
\end{tabular}

\vfill

\textsc{Facultad de Ingeniería - UBA}

\end{center}
\end{titlepage}

% --- Index ---
\tableofcontents
\newpage

% --- Sections ---
\section{Resumen}
% Aquí va el resumen en español.

\section{Palabras clave}
% Lista de palabras clave.

\section{Abstract}
% Aquí va el abstract en inglés.

\section{Keywords}
% Lista de keywords en inglés.

\section{Introducción}
% Introducción general del proyecto.

\section{Estado del Arte}
% Revisión de trabajos previos, tecnología existente, etc.

\section{Problema detectado y/o faltante}
% Qué problema se detectó, por qué es relevante, etc.

\section{Solución propuesta}
% Descripción de la propuesta para resolver el problema.

\section{Evaluación preliminar de impacto social y ambiental}
% Análisis del impacto potencial del proyecto.

\section{Metodología}
% Enfoque metodológico que se aplicará.

\section{Experimentación y/o validación}
% Cómo se probará la solución, experimentos, etc.

\section{Plan de actividades}
% Cronograma o planificación de tareas.

\section{Referencias}
% Bibliografía en formato apropiado (puede usarse BibTeX si se desea).

\section{Anexos}
% Cualquier material adicional, gráficas, códigos, etc.

\end{document}
