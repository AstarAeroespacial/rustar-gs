\section*{Solución propuesta}

La solución propuesta consiste en el desarrollo de un sistema de software integral para la operación de una estación terrena dedicada al control de satélites tipo CubeSat. Este sistema centralizará en una única plataforma todas las funciones necesarias para realizar tareas de telemetría, seguimiento y control (TT\&C), actualmente distribuidas en múltiples herramientas de uso manual.

El software estará compuesto por los siguientes módulos principales:

\begin{itemize}
    \item \textbf{Interfaz gráfica de usuario (GUI):} permitirá a los operadores interactuar de manera intuitiva con el sistema para visualizar datos, controlar el movimiento de las antenas y gestionar las comunicaciones con el satélite.
    
    \item \textbf{Módulo de seguimiento orbital:} calculará y actualizará en tiempo real la posición del satélite utilizando modelos orbitales, y comandará automáticamente la orientación de las antenas a través de motores o controladores físicos.
    
    \item \textbf{Módulo SDR (radio definida por software):} se encargará de la transmisión y recepción de señales mediante dispositivos SDR compatibles. Permitirá configurar parámetros de modulación, frecuencia y ganancia, integrándose con herramientas como GNU Radio.
    
    \item \textbf{Módulo de telemetría y comandos:} decodificará los datos recibidos, los almacenará en una base de datos y los presentará en tiempo real a través de la GUI. Además, gestionará el envío de comandos al satélite de acuerdo a protocolos específicos.
    
    \item \textbf{Base de datos:} almacenará telemetría histórica, registros de eventos y parámetros de configuración, permitiendo consultas eficientes y análisis posteriores.
\end{itemize}

El sistema será desarrollado utilizando \textbf{Rust} como lenguaje principal, debido a su alto rendimiento, garantías de seguridad en memoria y creciente adopción en sistemas embebidos y de control. Para el procesamiento de señales, se emplearán herramientas de código abierto como \textbf{GNU Radio}, a través de bindings o integración externa. También se utilizarán bibliotecas especializadas para el cálculo de efemérides orbitales y predicción de pases.

Las decisiones tecnológicas restantes se definirán en función de los siguientes criterios:

\begin{itemize}
    \item Compatibilidad con el hardware disponible (dispositivos SDR, motores, controladores).
    \item Facilidad de integración con bibliotecas existentes.
    \item Licenciamiento abierto y adecuación al entorno académico del proyecto.
    \item Soporte en plataformas Unix/Linux.
\end{itemize}

Esta solución buscará minimizar dependencias innecesarias y maximizar la modularidad del código para facilitar futuras ampliaciones o adaptaciones a otras misiones CubeSat.
