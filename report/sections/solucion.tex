El sistema centralizará en una única plataforma todas las funciones necesarias para realizar tareas de \textit{TT\&C} que actualmente se encuentran distribuidas en múltiples herramientas de uso manual.

El software estará compuesto por los siguientes módulos principales:

\begin{itemize}
    \item \textbf{Módulo de seguimiento orbital:} calculará y actualizará en tiempo real la posición del satélite a seguir utilizando modelos orbitales. También controlará automáticamente la orientación de las antenas a través del brazo robot de la estación terrena.
    
    \item \textbf{Módulo \textit{SDR}:} se encargará de la transmisión y recepción de señales mediante dispositivos \textit{SDR} compatibles. Permitirá configurar parámetros de modulación, frecuencia y ganancia, integrándose con herramientas como \textit{GNU Radio}.
    
    \item \textbf{Módulo de telemetría y comandos:} decodificará los datos recibidos, los almacenará en una base de datos y los presentará en tiempo real a través de la \textit{GUI}. Además, gestionará el envío de comandos al satélite de acuerdo a protocolos específicos.
    
    \item \textbf{Base de datos:} almacenará telemetría histórica, registros de eventos y parámetros de configuración, permitiendo consultas eficientes y análisis posteriores.
    
    \item \textbf{Interfaz gráfica de usuario (GUI):} permitirá a los operadores interactuar de manera intuitiva con el sistema para visualizar datos, controlar el movimiento de las antenas y gestionar las comunicaciones con el satélite.
\end{itemize}

El sistema será desarrollado utilizando \textbf{Rust} como lenguaje principal, debido a su alto rendimiento, garantías de seguridad en memoria y creciente adopción en sistemas embebidos y de control. Para el procesamiento de señales, se emplearán herramientas de código abierto como \textbf{\textit{GNU Radio}}, a través de bindings o integración externa. También se utilizarán bibliotecas especializadas para el cálculo de efemérides orbitales y predicción de pases.

Las decisiones tecnológicas restantes se definirán en función de los siguientes criterios:

\begin{itemize}
    \item Compatibilidad con el hardware disponible en el laboratorio.
    \item Facilidad de integración con bibliotecas existentes.
    \item Licenciamiento abierto y adecuación al entorno académico del proyecto.
    \item Soporte en plataformas Unix/Linux.
\end{itemize}

Esta solución buscará minimizar dependencias innecesarias y maximizar la modularidad del código para facilitar futuras ampliaciones o adaptaciones a otras misiones \textit{CubeSat}.
