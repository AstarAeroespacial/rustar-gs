% Los proyectos de ingeniería, incluyendo los de Ingeniería en Informática, suelen tener impactos económicos, sociales y/o ambientales. En la propuesta de Trabajo Profesional se debe incluir un análisis preliminar de impacto, de carácter descriptivo solamente, y con los conocimientos que se tengan en el momento de realizar la misma.

Evaluación preliminar de impacto económico, social y ambiental.

Impacto económico

El desarrollo de un sistema unificado de control para estaciones terrenas representa una reducción significativa de costos operativos. Actualmente, se requiere la utilización de múltiples programas de propósito general (como GNURadio, Orbitron, etc.), lo cual implica tiempo de configuración, entrenamiento y mantenimiento. La integración de todas las funcionalidades en una sola plataforma de código abierto permitirá reducir estos costos, facilitar futuras actualizaciones y minimizando la dependencia de software propietario o de licencias comerciales restrictivas.

Además, el sistema desarrollado podría ser reutilizado o adaptado por otras instituciones o proyectos académicos, favoreciendo la economía de recursos en desarrollos similares.

Impacto social

La implementación de esta herramienta contribuirá al fortalecimiento del ecosistema espacial nacional, fomentando la formación de profesionales capacitados en tecnologías aeroespaciales y sistemas embebidos. Asimismo, facilitará el acceso a tecnologías complejas mediante interfaces simples y personalizables, permitiendo su uso en entornos educativos y de investigación.

El proyecto también promueve la colaboración entre distintas disciplinas (electrónica, informática, aeroespacial) y puede ser una base para trabajos futuros en satélites educativos, comunicaciones de emergencia o monitoreo ambiental desde el espacio.

El desarrollo de este proyecto es parte fundamental del proyecto Astar, que es uno de los primeros de su clase en la Argentina. El desarrollo de un CubeSat propio de la Universidad de Buenos Aires puede traer varios beneficios académicos y tecnológicos, y ayudan a situar a la institución a la vanguardia del desarrollo tecnológico.

Impacto ambiental

Desde el punto de vista ambiental, el sistema tiene un impacto directo reducido, ya que se trata de software. Sin embargo, su utilización eficiente puede optimizar la operación de estaciones terrenas, evitando consumos innecesarios de energía durante los periodos inactivos o mal configurados. A largo plazo, el uso de esta herramienta puede contribuir a una mejor planificación de misiones satelitales con fines ambientales, como el monitoreo climático, la detección de incendios o la gestión de recursos naturales.
