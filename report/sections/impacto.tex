\section*{Impacto económico, social y ambiental}

\subsection*{Impacto económico}

El desarrollo de un sistema unificado de control para estaciones terrenas implica una reducción significativa en los costos operativos. En la estructura actual, es necesario recurrir a múltiples programas (como GNURadio, Orbitron, entre otros), lo que demanda tiempo de configuración, entrenamiento y mantenimiento. La centralización de estas funcionalidades en una única plataforma de código abierto permitirá disminuir estos costos, facilitar futuras actualizaciones y minimizar la dependencia de software propietario o licencias comerciales restrictivas.

Asimismo, el sistema podrá ser reutilizado o adaptado por otras instituciones educativas o proyectos de investigación, favoreciendo la economía de recursos en desarrollos similares.

\subsection*{Impacto social}

La implementación de esta herramienta contribuirá al fortalecimiento del ecosistema espacial nacional, promoviendo el interés y formación de profesionales capacitados en tecnologías aeroespaciales y sistemas embebidos. Además, permitirá el acceso a tecnologías complejas a través de interfaces simples y personalizables, facilitando su uso en entornos educativos, académicos y de investigación.

El proyecto fomenta la colaboración interdisciplinaria entre áreas como la ingeniería electrónica, la informática y la aeroespacial, y puede sentar las bases para futuras iniciativas en satélites educativos, comunicaciones de emergencia o monitoreo ambiental desde el espacio.

Este desarrollo forma parte del proyecto \textit{Astar}, uno de los pioneros en su tipo en la Argentina. La construcción de un \textit{CubeSat} por parte de la Universidad de Buenos Aires no solo representa un avance académico y tecnológico significativo, sino que también posiciona a la institución como referente en el desarrollo de tecnología espacial en el país.

\subsection*{Impacto ambiental}

Desde el punto de vista ambiental, el impacto directo del sistema es acotado, al tratarse exclusivamente de software. No obstante, su uso eficiente puede optimizar la operación de las estaciones terrenas, reduciendo consumos energéticos innecesarios durante períodos de inactividad o mala configuración. A largo plazo, esta herramienta podría contribuir indirectamente a una mejor planificación de misiones satelitales con objetivos ambientales, como el monitoreo climático, la detección temprana de incendios o la gestión de recursos naturales desde el espacio.
