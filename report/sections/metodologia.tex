% Esta sección se usarpa para explicitar la metodología general de trabajo en el proyecto, incluyendo los roles de los tutores y los estudiantes.
% Como el proyecto va a incluir el desarrollo de un producto, sea de software o un sistema de software-hardware (sistemas embebidos o ciberfísicos), se debe usar esta sección también para especificar el proceso de desarrollo del producto.
% Si hay cuestiones metodológicas no definidas, explicitarlo y explicar brevemente de qué depende la decisión.
% Como todo proyecto tiene riesgos, deberá haber una lista de los riesgos iniciales del proyecto.

\section*{Metodología de trabajo}

El proyecto se abordará mediante una metodología iterativa e incremental, inspirada en principios de desarrollo ágil. Esta metodología permitirá validar continuamente tanto los avances técnicos como las decisiones de diseño, adaptándose a los cambios que puedan surgir durante la integración con el hardware real o la evolución de los requerimientos funcionales.

El trabajo se dividirá en fases, con entregas parciales de componentes funcionales. Cada iteración incluirá etapas de análisis, diseño, desarrollo, prueba e integración, fomentando la retroalimentación constante y la mejora progresiva del sistema.

\subsection*{Roles}

Las personas involucradas directamente con el proyecto pueden ser divididas en los siguientes roles:

\begin{itemize}
    \item \textit{Estudiantes desarrolladores:} serán responsables de la ingeniería del sistema, incluyendo el diseño arquitectónico, codificación, pruebas, documentación técnica y coordinación de la integración con los distintos dispositivos físicos (antenas, radios SDR, motores, etc.).
    
    \item \textit{Tutores académicos:} brindarán acompañamiento metodológico y técnico. Supervisarán las decisiones de diseño, las tecnologías seleccionadas y el cumplimiento de los objetivos del proyecto Astar mediante reuniones periódicas y revisión de entregas.
    
    \item \textit{Tutores técnicos (miembros del proyecto Astar):} colaborarán de forma puntual con conocimiento experto en hardware, comunicaciones satelitales, protocolos específicos o integración con otros subsistemas del CubeSat. También podrán validar aspectos prácticos de la experiencia de usuario del software.
\end{itemize}
    
\subsection*{Proceso de desarrollo}

El sistema será desarrollado en Rust, aplicando principios de diseño modular, separación de responsabilidades y reutilización de componentes. Los principales módulos (GUI, seguimiento orbital, telemetría y comandos, base de datos, SDR, etc.) se diseñarán como componentes acoplados de forma débil, facilitando su desarrollo en paralelo y su posterior integración. Como es necesaria la integración del sistema con varios componentes de hardware (receptores y emisores de radiofrecuencia, motores de control de la antena, etc.), se procurará que la mayor parte de los componentes sean agnósticos a los detalles específicos de cada modelo, para procurar la mayor reutilizabilidad posible de componentes de software con diversas configuraciones de hardware subyacentes.

Se utilizarán herramientas como:

\begin{itemize}
    \item Git para control de versiones.
    \item Tableros de seguimiento para la planificación de tareas e hitos.
    \item Documentación técnica continua utilizando herramientas de generación automática desde el código.
    \item Pruebas unitarias y funcionales sobre simuladores antes de la prueba con hardware real.
    \item Features propias del lenguaje para la modularización, manejo de dependencias, y desarrollo de pruebas.
    \item Ensayos y pruebas de integración con hardware real cuando los sistemas estén listos.
    \item Integración y entrega continua para garantizar la calidad y funcionamiento del código en todo momento.
\end{itemize}

\subsection*{Riesgos iniciales identificados}

Dado el contexto del desarrollo del proyecto, identificamos los siguientes riesgos que podrían perjudicar el desarrollo del mismo:

\begin{itemize}
    \item \textit{Integración con hardware SDR:} las bibliotecas nativas en Rust son limitadas. Podría ser necesario usar FFI hacia librerías en C/C++ o desarrollar bindings propios, lo cual añade complejidad.
    
    \item \textit{Disponibilidad de hardware:} el acceso limitado o demorado a componentes clave de la estación terrena podría afectar los tiempos de prueba e integración.

    \item \textit{Validación en entorno real:} la posibilidad de testear funcionalidades con satélites en órbita depende de ventanas de tiempo reducidas (pasos orbitales), lo que limita las oportunidades de validación en condiciones reales.

    \item \textit{Falta de recursos y herramientas existentes:} según lo investigado por el equipo, actualmente no existen muchas librerías ni proyectos relacionados al procesamiento de señales en Rust, y los existentes están aún en etapas tempranas de desarrollo, por lo que es más probable que tengamos que implementar manualmente funcionalidades que no están soportadas o no funcionan bien en las librerías disponibles.
    
    \item \textit{Dependencia del equipo de Astar:} la colaboración fluida con el equipo del proyecto satelital es crítica. Problemas de comunicación, solapamiento de actividades o prioridades distintas podrían afectar el avance del proyecto.
\end{itemize}