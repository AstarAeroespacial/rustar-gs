\section*{Metodología de trabajo}

El proyecto se abordará de una forma iterativa e incremental, inspirada en los principios de las metodologías ágiles. Esto permitirá validar continuamente tanto los avances técnicos específicos como las decisiones de diseño, adaptándose a los cambios que puedan surgir durante la integración con el hardware real o la evolución de los requerimientos funcionales por parte del LABi.

El trabajo se dividirá en fases, con entregas parciales de módulos funcionales. Cada iteración incluirá etapas de análisis e investigación, diseño, desarrollo, prueba e integración, fomentando la retroalimentación constante y la mejora progresiva del sistema.

\subsection*{Roles}

Las personas involucradas en el proyecto tendrán los siguientes roles:

\begin{itemize}
    \item \textbf{Estudiantes desarrolladores:} serán responsables de la ingeniería del sistema, incluyendo el diseño de la arquitectura, codificación, pruebas, documentación técnica y coordinación de la integración con los distintos dispositivos físicos (antenas, motores, \textit{SDR}, etc.).
    
    \item \textbf{Tutores académicos:} brindarán acompañamiento metodológico y técnico en general. Supervisarán las decisiones de diseño, las tecnologías seleccionadas y el cumplimiento de los objetivos del Trabajo Profesional mediante reuniones periódicas y revisión de entregas.
    
    \item \textbf{Tutores técnicos (miembros del proyecto Astar):} colaborarán de forma puntual con conocimiento experto en hardware, comunicaciones satelitales, protocolos específicos e integración con otros subsistemas del \textit{CubeSat}. También podrán validar aspectos prácticos de la experiencia de usuario del software.
\end{itemize}
    
\subsection*{Proceso de desarrollo}

Durante el proceso de desarrollo del sistema se aplicarán principios de diseño modular, separación de responsabilidades y reutilización de componentes. Los principales módulos se diseñarán como componentes acoplados de forma débil, facilitando su desarrollo en paralelo y su posterior integración. Como es necesaria la integración del sistema con varios componentes de hardware (receptores y emisores de radiofrecuencia, motores de control de la antena, etc.), se procurará que la mayor parte de los componentes sean agnósticos a los detalles específicos de cada modelo, para procurar la mayor reutilizabilidad posible de componentes de software con diversas configuraciones de hardware subyacentes.

Se utilizarán las siguientes herramientas:

\begin{itemize}
    \item \textbf{Recopilación de proyectos de referencia e información general:} Notion.
    \item \textbf{Control de versiones:} Git y GitHub.
    \item \textbf{Gestión de tareas e hitos:} Github Projects.
    \item \textbf{Automatización y CI:} GitHub Actions.
    \item \textbf{Documentación técnica:} Markdown y LaTeX.
    \item \textbf{Documentación del código:} \texttt{cargo doc}.
    \item \textbf{Seguimiento de errores y bugs:} GitHub Issues.
\end{itemize}

\subsection*{Riesgos iniciales identificados}

Dado el contexto del proyecto, identificamos los siguientes riesgos que podrían perjudicar y/o ralentizar el desarrollo del mismo:

\begin{itemize}
    \item \textbf{Integración con hardware \textit{SDR}:} las bibliotecas nativas en Rust son limitadas. Podría ser necesario usar una FFI (Foreign Function Interface) hacia librerías en C/C++ o desarrollar bindings propios, lo cual añade complejidad.
    
    \item \textbf{Disponibilidad de hardware:} el acceso limitado o demorado a componentes clave de la estación terrena podría afectar los tiempos de prueba e integración.

    \item \textbf{Validación en entorno real:} la posibilidad de testear funcionalidades con satélites en órbita depende de ventanas de tiempo reducidas (pasos orbitales), lo que limita las oportunidades de validación en condiciones reales.

    \item \textbf{Falta de recursos y herramientas existentes:} según lo investigado por el equipo, actualmente no existen muchas librerías ni proyectos relacionados al procesamiento de señales en Rust, y los existentes están aún en etapas tempranas de desarrollo, por lo que es más probable que tengamos que implementar manualmente funcionalidades que no están soportadas o no funcionan bien en las librerías disponibles.
    
    \item \textbf{Dependencia del equipo de Astar:} la colaboración fluida con el equipo del proyecto satelital es fundamental. Problemas de comunicación, solapamiento de actividades o distintas prioridades podrían afectar el regular avance del proyecto.
\end{itemize}