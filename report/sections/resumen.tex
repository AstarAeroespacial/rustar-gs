Este proyecto tiene como objetivo el desarrollo del software para una estación terrena dedicada a tareas de telemetría, seguimiento y control (TT\&C) de satélites de tipo CubeSat, desarrollados en el marco del proyecto Astar de la Facultad de Ingeniería de la Universidad de Buenos Aires. La estación terrena permitirá la comunicación con los satélites mediante dispositivos de radio definida por software (SDR), debiendo implementar las modulaciones y demodulaciones necesarias para una comunicación efectiva. El sistema deberá incorporar funciones para el seguimiento del satélite en tiempo real, permitiendo así que el sistema de rotación de la antena apunte correctamente al satélite durante su paso orbital.

Para garantizar el intercambio de datos de forma segura, eficiente y confiable, se implementará un protocolo de comunicaciones adecuado. El sistema deberá ser accesible de forma remota, dada la posibilidad de que los operadores y la estación se encuentren en ubicaciones distintas. Incluirá una interfaz gráfica de usuario (GUI) para facilitar la visualización del estado de la misión y el envío de comandos. El software será diseñado con un enfoque modular, extensible y basado en herramientas de código abierto, permitiendo su adaptación a distintas configuraciones y necesidades.

El objetivo es ofrecer una plataforma integral que optimice las operaciones de los operadores y facilite la interacción con el satélite. Además, al estar vinculado con el entorno académico, el sistema contribuirá al fortalecimiento de la formación práctica en el área aeroespacial, sirviendo como herramienta de aprendizaje y experimentación para estudiantes y docentes de la universidad.
