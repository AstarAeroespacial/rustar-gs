% En esta sección se debe presentar el problema detectado y/o oportunidad de mejora relacionada con la construcción de la aplicación presentada. La misma debe ser redactada de manera clara y concisa debe ser entendida en la primera lectura, este ítem es de fundamental importancia al momento de ser evaluado el proyecto.

Actualmente, el proyecto Astar cuenta con una estación terrena compuesta de una antena que puede posicionarse para seguir la órbita de satélites conocidos, y recibir datos de estos mediante un dispositivo de SDR, específicamente el RTL-SDR. Esta configuración sólo permite la recepción de señales. Es necesario un conjunto de programas externos para el control de la antena, y la recepción y demodulación de señales. Este enfoque fragmentado genera varias dificultades:

\begin{itemize}
    \item \textit{Complejidad operativa:} Los operadores deben interactuar con múltiples sistemas que no están integrados, lo que aumenta la carga cognitiva y reduce la eficiencia operativa.

    \item \textit{Falta de integración de funcionalidades:} Las distintas herramientas no se comunican entre sí de forma automática, lo que limita la capacidad de realizar acciones coordinadas y optimizadas.

    \item \textit{Limitación de las posibilidades: }
\end{itemize}

Además, la configuración actual no permite la transmisión de señales, por lo cual es imposible el control de una futura misión satelital.

Estas limitaciones dificultan la operación de la estación terrena, ralentizan el proceso de toma de decisiones y limitan el potencial de la misión satelital. Así, se presenta una oportunidad de mejora al centralizar todas las funcionalidades en una sola plataforma unificada, simplificando el flujo de trabajo, reduciendo la complejidad y mejorando la eficiencia operativa. El objetivo de este trabajo es entonces desarrollar un sistema integrado que provea una interfaz sencilla para llevar a cabo todas las tareas relacionadas con la telemetría y control del satélite.
