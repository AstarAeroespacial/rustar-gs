% En esta sección se hace un breve mapeo documental de los avances relacionados a la aplicación a desarrollar y es la base para demostrar la validez del proyecto. Aquí se remarca que, aunque es un trabajo profesional, el mismo debe ser innovador y encontrarte en el estado del arte de la ingeniería del software.

Herramientas actualmente utilizadas en estaciones terrenas:

GNU Radio: herramienta poderosa de radio definida por software (SDR), flexible pero con curva de aprendizaje alta, sin interfaz específica para TT\&C.

Orbitron o GPredict: software para predicción y seguimiento satelital, útil pero no integrable directamente con procesamiento de señales o bases de datos de telemetría.

SatNOGS: red global y software para estaciones terrenas open-source. Aunque es avanzado y automatizado, está pensado más para redes comunitarias que para operaciones específicas de CubeSats en entornos académicos.

Limitaciones de las soluciones actuales
No existe una herramienta única que integre todas las funcionalidades necesarias (seguimiento orbital, SDR, visualización, control).

Requieren operación manual, conocimientos técnicos avanzados y coordinación entre varios programas.

Algunas no están diseñadas para integrarse fácilmente con hardware específico o necesidades académicas locales.

Propuesta de innovación

Este proyecto busca llenar ese vacío desarrollando una solución integrada, modular y de fácil uso, centrada en las necesidades del proyecto Astar. La interfaz gráfica, la integración directa con el hardware y la automatización del flujo de trabajo representan una mejora sustancial frente a las herramientas existentes, adaptada al contexto educativo y experimental.

