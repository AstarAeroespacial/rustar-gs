Actualmente, en ambientes y organizaciones académicas y de la industria se utilizan diversas herramientas para \textit{TT\&C} en el ámbito de los \textit{CubeSats}, por ejemplo:

\begin{itemize}
    \item \textit{GNU Radio}: herramienta de \textit{SDR} altamente flexible y extensible, pero con una curva de aprendizaje pronunciada y sin una interfaz orientada específicamente a operaciones \textit{TT\&C}.
    
    \item \textit{Orbitron} y \textit{GPredict}: aplicaciones utilizadas para predicción y seguimiento satelital, útiles para el cálculo de efemérides y visualización de órbitas, pero no integrables directamente con procesamiento de señales ni con sistemas de gestión de telemetría.
    
    \item \textit{SatNOGS}: red global y conjunto de herramientas open-source para estaciones terrenas automatizadas. Aunque ofrece una infraestructura avanzada, está más enfocada en redes colaborativas y recopilación pasiva de datos, lo cual limita su aplicabilidad para operaciones dedicadas y control activo de \textit{CubeSats} en entornos académicos.
\end{itemize}
