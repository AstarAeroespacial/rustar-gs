% Breve reseña de las partes con las que cuenta el presente documento y se presenta el problema y/o servicio a mejorar y/o resolver. Punto de partida para presentar el proyecto, se debe ser preciso y enfocado en el objetivo del proyecto.

El presente documento describe el desarrollo de un sistema de software para una estación terrena dedicada a tareas de telemetría, seguimiento y control (TT\&C) de satélites tipo CubeSat. Este trabajo se enmarca en el proyecto Astar, una iniciativa del Laboratorio Abierto (LABi) de la Facultad de Ingeniería de la Universidad de Buenos Aires orientada a la formación práctica y la investigación en el ámbito aeroespacial.

Actualmente, la estación terrena del proyecto Astar depende de múltiples herramientas de software independientes para su operación. Cada una de estas herramientas —como GNU Radio para el procesamiento de señales u Orbitron para el control de las antenas— debe ser operada de forma manual, lo que fragmenta el flujo de trabajo, incrementa la complejidad operativa y dificulta la formación de nuevos usuarios.

Ante esta situación, el objetivo del presente proyecto es desarrollar una aplicación integral que unifique en una sola plataforma todas las funcionalidades necesarias para operar la estación terrena. Esta solución busca centralizar y simplificar las tareas de seguimiento, recepción y transmisión de datos, facilitando así el control de las misiones satelitales y mejorando la experiencia de los operadores.

El sistema estará basado en tecnologías de código abierto y será desarrollado con un enfoque modular y extensible, permitiendo su adaptación a diferentes configuraciones de hardware. Además, contará con una interfaz gráfica amigable, diseñada para brindar una experiencia clara e intuitiva a estudiantes, docentes e investigadores.