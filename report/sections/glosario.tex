\subsection{Glosario}
\textit{Analog-Digital Converter (ADC):} Conversor analógico-digital. Es un dispositivo de hardware que transforma señales analógicas a un formato digital (binario).

\textit{CubeSat:} Satélite pequeño, compuesto de unidades cúbicas de 10 cm de lado y de masa menor a 2 kg, generalmente construidos con componentes estándar de mercado. Definido en el estándar ISO 17770:2017.

% \textit{Digital-Analog Converter (DAC):} Conversor digital-analógico. Es un dispositivo que transforma señales representadas en formato digital (binario) a su equivalente analógico.

\textit{Estación terrena:} Infraestructura terrestre equipada para comunicarse con satélites mediante enlaces de radiofrecuencia. Se encarga de transmitir comandos, recibir datos de telemetría y rastrear la trayectoria orbital de los satélites.

% \textit{Field Programmable Gate Array (FPGA):} Circuito integrado con la capacidad de ser reprogramado varias veces de forma sencilla. Pueden ser configurados para realizar distintas funciones lógicas y permiten el desarrollo de hardware y software de forma simultánea.

% \textit{In-Phase and Quadrature components (IQ):} Representación de una señal sinusoidal como suma de 2 señales ortogonales ('en fase' y 'cuadratura'). Permite representar una señal analógica utilizando pares de números.

\textit{Software Defined Radio (SDR):} Radio definida por software. Se trata de un sistema compuesto de un receptor de radiofrecuencia, un conversor análogico-digital y un programa de computadora. Este último se encarga de realizar por medio de software diversos tipos de procesamiento sobre la señal que antiguamente se solían hacer por medio de hardware.

\textit{TT\&C (Telemetría, Seguimiento y Control):} conjunto de funciones esenciales en operaciones satelitales que permite monitorear el estado del satélite (telemetría), calcular y predecir su posición (seguimiento), y enviarle comandos desde la estación terrena (control).

% \textit{Two Line Element (TLE):} Formato estándar para la codificación de información sobre objetos en órbita. Utilizando la data contenida en un TLE, se puede calcular la velocidad y posición de un satélite en cualquier instante.


% Glossary

% \textit{Analog-Digital Converter:} A hardware device that transforms analog signals into a digital \( binary \) format.

% \textit{CubeSat:} Small satellite composed of cubic units of 10 cm per side and a mass under 2 kg, generally built using off-the-shelf components. Defined by ISO 17770:2017.

% \textit{Digital-Analog Converter:} A device that converts digitally represented (binary) signals into their analog equivalent.

% \textit{Field Programmable Gate Array:} An integrated circuit that can be reprogrammed multiple times easily. It can be configured to perform different logical functions and enables concurrent hardware and software development.

% \textit{Ground Station:} Ground-based infrastructure equipped to communicate with satellites via radio frequency links. It is responsible for sending commands, receiving telemetry data, and tracking satellites' orbital paths.

% \textit{In-Phase and Quadrature components:} A representation of a sinusoidal signal as the sum of two orthogonal signals ("in-phase" and "quadrature"). It allows analog signals to be represented using pairs of numbers.

% \textit{Software Defined Radio:} A radio system composed of a radiofrequency front-end, an analog-digital converter, and computer software that performs signal processing tasks traditionally done in hardware.

% \textit{TT\&C (Telemetry, Tracking and Command):} A set of essential functions in satellite operations that allow monitoring the satellite's state (telemetry), calculating and predicting its position (tracking), and sending it commands from the ground station (command).

% \textit{Two Line Element:} A standard format for encoding information about orbital objects. Using data from a TLE, a satellite’s position and velocity can be calculated at any given moment.