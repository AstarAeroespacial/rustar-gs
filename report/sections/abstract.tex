This project aims to develop the software for a ground station dedicated to telemetry, tracking, and command (TT\&C) tasks for CubeSat-type satellites, developed within the framework of the Astar project at the Faculty of Engineering of the University of Buenos Aires. The ground station will enable communication with the satellites through software-defined radio (SDR) devices, requiring the implementation of the necessary modulations and demodulations for signal transmission. The system will incorporate real-time satellite tracking functions, allowing the antenna rotation system to correctly point to the satellite during its orbital pass.

To ensure secure, efficient, and reliable data exchange, an appropriate communication protocol will be implemented. The system will be remotely accessible, considering that the station and mission operators may be in different locations. It will include a graphical user interface (GUI) to facilitate mission status visualization and command transmission. The software will be designed with a modular and extensible approach, based on open-source tools, allowing adaptation to different configurations and requirements.

The objective is to provide a comprehensive platform that optimizes operator tasks and facilitates satellite interaction. Furthermore, as part of an academic environment, the system will contribute to strengthening practical training in the aerospace field, serving as a learning and experimentation tool for students and faculty at the university.
